\documentclass[../main/main.tex]{subfiles}
\begin{document}
\chapter*{Conclusions}
\addcontentsline{toc}{chapter}{Conclusions}
In this thesis, we presented an application of quantum optimal control to one- and two-qubit gates based on superconducting qubits.\\
We presented at first the theory of superconductivity, focusing on the microscopical theory that explains the superconductivity phenomena. We highlighted two important concepts of the theory: Cooper pairs and superconducting energy gap.\\
Then, we reviewed how macroscopic quantum effects manifest themselves in Josephson junctions at low temperatures. In particular, the tunneling of Cooper pairs across the barrier maintains the coherence of the pairs' wave function at temperatures below the critical one and for setups with characteristic energies below the superconducting energy gap. We showed which experimental conditions are required to implement a quantum computing device based on superconducting Josephson junctions. The implementation of a charge qubit was presented in detail, while a possible improvement of the setup was mentioned. With this improvement, we can handle both x- and z-rotations of the charge qubit states, allowing us to realize all the one-qubit gates. Moreover, a possible solution to couple two qubits was briefly mentioned. Then, we presented how to obtain in practice the most common one-qubit gates and the CNOT gate by tuning charge qubit controls and by setting evolution times accordingly.\\
An improvement of the qubit gates implementation based on quantum optimal control was discussed. We described the theory of quantum optimal control, focusing on the Chopped RAndom Basis algorithm, and we applied such theory to the optimization of qubit gates. We defined the cost function to be minimized and we explained its connection to the gate infidelity. In other words, we showed how to estimate numerically the infidelity of a gate. We explained in details the minimization algorithm that was used in the optimization. The infidelity of a NOT gate was reduced by ten orders of magnitude by adding an additional control term in the Hamiltonian of the qubit. In other terms, the cost function of the NOT gate decreased by a factor of $10^{10}$. Similarly, the CNOT gate infidelity was reduced by six orders of magnitude thanks to the optimized control pulses.\\
In conclusion, quantum optimal control can, theoretically, enhance the performances of existing gate implementations using software-based solutions. In our optimizations we achieved great improvements in the performances. However, these optimizations are only a first step towards the optimization of more complex systems, described by more general equations and disturbed by more general external dephasing sources, until we will be able to optimize real-world quantum systems.

\end{document}