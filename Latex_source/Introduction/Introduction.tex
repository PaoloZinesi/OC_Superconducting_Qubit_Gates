\documentclass[../main/main.tex]{subfiles}
\begin{document}
\chapter*{Introduction}
\addcontentsline{toc}{chapter}{Introduction}

Richard Feynman in 1982 introduced the idea of a quantum computer, a new kind of computer that can simulate quantum systems efficiently \cite{Feynman1982}. Feynman's work builds from the fact that the exact simulation of quantum systems with universal classical computers is not always possible. In fact, a theory which reproduces quantum mechanical predictions needs to have a non-local structure \cite{PhysicsPhysiqueFizika.1.195}. Classical computers follow the rules of classical physics, which is a local theory, while quantum physics admit the existence of non-local correlations between entangled particles. So, we need computers that follows quantum mechanics to simulate all the possible quantum systems.\\
Since then, quantum technologies progressed considerably and a variety of dedicated software and hardware have been developed. Researchers found out that quantum mechanical properties, such as superposition and entanglement, can speed up some expensive tasks in certain algorithms. As an example, the Quantum Fourier Transform (QFT) algorithm \cite{coppersmith2002approximate} computes the discrete Fourier transform in polynomial time with respect to the input size, while the classical Fast Fourier Transform (FFT) algorithm needs exponential time.\\
The physical realization of a platform that can run such algorithms is one of the main challenges that quantum technologies have to overcome. Experimental imperfections and, eventually, interactions with the environment affect the state of the system, giving rise to errors. The quality of the quantum hardware depends on both the number of qubits, the two-state quantum systems, and the error rate of the setup.\\
In this thesis we concentrate on superconducting qubits as a possible implementation of quantum computers. The superconducting qubits are based on devices called Josephson junctions, in which pairs of electrons tunnel across a thin insulating barrier that separates two superconductors. Under suitable conditions, the Josephson junction reduces to a two-state system that can be used for quantum computation \cite{RevModPhys.73.357}. Operations on single qubits are performed by tuning voltages and magnetic fields and the coupling between two qubits provides the necessary two-qubit operations.\\
Enterprises such as Google \cite{Castelvecchi2017google}, IBM \cite{Castelvecchi2017ibm}, Rigetti \cite{rigetti_computing_2021}, are now developing commercial processors with more than 50 qubits using superconducting circuits. These processors allow to implement universal quantum computation and Google claimed in 2019 to have reached quantum supremacy using a \mbox{53-qubit} processor \cite{google48651}. Nevertheless, D-Wave announced in 2019 a 5000-qubit processor \cite{dwave_5000q} specialized in quantum annealing but unable to support generic quantum computation. The results obtained by the scientific community attract investments and make the superconducting technologies a flourishing research field.\\
One important part of the experimental research work is focused on the construction of apparatus that are more robust against external dephasing sources. This goal can be achieved, for instance, by improving electromagnetic shielding or the cryogenic apparatus. Another solution to reduce dephasing effects in the processor comes from mathematical optimization. Quantum Optimal Control (QOC) \cite{Werschnik_2007,Glaser2015,boscain2020introduction} uses the tools of Optimal Control Theory to find the time-dependent control pulses that minimize a specific cost function $J$. That function $J$ depends on the goal of the optimization. The optimal control pulses allow to obtain gates that are less influenced by environmental noise and experimental imperfections.\\

In the first chapter of the thesis we introduce the theory of superconductivity, starting from the early experimental observations and presenting the empirical London equations to explain them. Then, we present the basic concepts of the BCS theory and we describe the characteristics of the BCS ground state. From these results we present a derivation of the second London equation.\\
In the second chapter we describe how the theory of superconductivity enables the construction of superconducting qubits using Josephson junctions. A simple theoretical explanation of the Josephson effect is presented. We then describe the properties of charge qubits and a possible real-world implementation of qubit gates.\\
Finally, in the third chapter we add an external noise source to the implementation of the second chapter and we observe the deviations of the gates from the expected behavior. Then, we use Quantum Optimal Control to implement gates that are robust against external noise sources.\\




\end{document}
